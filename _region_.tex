\message{ !name(thesis.tex)}%%%%%%%%%%%%%%%%%%%%%%%%%%%%%%%%%%%%%%%%%%%%%%%%%%%%%%%%%%%%%%%%%%%%
%% I, the copyright holder of this work, release this work into the
%% public domain. This applies worldwide. In some countries this may
%% not be legally possible; if so: I grant anyone the right to use
%% this work for any purpose, without any conditions, unless such
%% conditions are required by law.
%%%%%%%%%%%%%%%%%%%%%%%%%%%%%%%%%%%%%%%%%%%%%%%%%%%%%%%%%%%%%%%%%%%%

\documentclass[
  digital, %% This option enables the default options for the
           %% digital version of a document. Replace with `printed`
           %% to enable the default options for the printed version
           %% of a document.
  table,   %% Causes the coloring of tables. Replace with `notable`
           %% to restore plain tables.
  lof,     %% Prints the List of Figures. Replace with `nolof` to
           %% hide the List of Figures.
  lot,     %% Prints the List of Tables. Replace with `nolot` to
           %% hide the List of Tables.
  %% More options are listed in the user guide at
  %% <http://mirrors.ctan.org/macros/latex/contrib/fithesis/guide/mu/fi.pdf>.
]{fithesis3}
%% The following section sets up the locales used in the thesis.
\usepackage[resetfonts]{cmap} %% We need to load the T2A font encoding
\usepackage[T1,T2A]{fontenc}  %% to use the Cyrillic fonts with Russian texts.
\usepackage[
  main=english, %% By using `czech` or `slovak` as the main locale
                %% instead of `english`, you can typeset the thesis
                %% in either Czech or Slovak, respectively.
  german, russian, czech, slovak %% The additional keys allow
]{babel}        %% foreign texts to be typeset as follows:
%%
%%   \begin{otherlanguage}{german}  ... \end{otherlanguage}
%%   \begin{otherlanguage}{russian} ... \end{otherlanguage}
%%   \begin{otherlanguage}{czech}   ... \end{otherlanguage}
%%   \begin{otherlanguage}{slovak}  ... \end{otherlanguage}
%%
%% For non-Latin scripts, it may be necessary to load additional
%% fonts:
\usepackage{paratype}
\def\textrussian#1{{\usefont{T2A}{PTSerif-TLF}{m}{rm}#1}}
%%
%% The following section sets up the metadata of the thesis.
\thesissetup{
    date          = \the\year/\the\month/\the\day,
    university    = mu,
    faculty       = fi,
    type          = mgr,
    author        = David Kouřil,
    gender        = m,
    advisor       = Ivan Viola,
    title         = {Maya2CellVIEW: 3D Package Integrated Tool for Creating Large and Complex Molecular Scenes},
    TeXtitle      = {Maya2CellVIEW: 3D Package Integrated Tool for Creating Large and Complex Molecular Scenes},
    keywords      = {molecular visualization, illustration, Maya, Unity, ...},
    TeXkeywords   = {keyword1, keyword2, \ldots},
}
\thesislong{abstract}{
    Scientific illustrators communicate the cutting edge of research through their illustrations. There are numerous software tools that assist them with this job. The aim of this thesis is to push abilities of illustrators working on a large scale molecular scenes. This is done by connecting two software packages - Maya and cellVIEW - combining the rendering possibilities of cellVIEW and modeling tools of Maya which results in more effective and efficient workflow.
}
\thesislong{thanks}{
    This is the acknowledgement for my thesis, which can

    span multiple paragraphs.
}
%% The following section sets up the bibliography.
\usepackage{csquotes}
\usepackage[              %% When typesetting the bibliography, the
  backend=biber,          %% `numeric` style will be used for the
  style=numeric,          %% entries and the `numeric-comp` style
  citestyle=numeric-comp, %% for the references to the entries. The
  sorting=none,           %% entries will be sorted in cite order.
  sortlocale=auto         %% For more unformation about the available
]{biblatex}               %% `style`s and `citestyles`, see:
%% <http://mirrors.ctan.org/macros/latex/contrib/biblatex/doc/biblatex.pdf>.
\addbibresource{example.bib} %% The bibliograpic database within
                          %% the file `example.bib` will be used.
\usepackage{makeidx}      %% The `makeidx` package contains
\makeindex                %% helper commands for index typesetting.
%% These additional packages are used within the document:
\usepackage{paralist}
\usepackage{amsmath}
\usepackage{amsthm}
\usepackage{amsfonts}
\usepackage{url}
\usepackage{menukeys}
\begin{document}

\message{ !name(thesis.tex) !offset(17) }
\chapter{State of the art
I think the sections could be: 3D Modeling Software (Maya, C4D), Niche (Specific) Tools (cellVIEW, ...), Data Generation (cellPack)
Popsat: Molecular Flipbook, Molecular workbench, molecular maya, mcell, cellblender, pymol, VMD, ePMV(scripps)
In this chapter we will introduce the software that is available for scientific illustrators to use now. Illustrators have gone a long way from doing these works by hand and nowadays there exist a variety of programs that help them do their job communicating the scientific findings.
First and foremost, the pipeline of almost all of these illustrators is strongly based on some 3D modeling software. The biggest players right now are Autodesk Maya, Cinema 4D, Autodesk 3Ds Max Design, Blender or even up and comming MODO(https://www.thefoundry.co.uk/case-studies/viscira/), first two named being the dominating choices between illustrators. These programs are complex and aim to provide tool for all kinds of users. Their main functionality revolves around two activities - creation of the 3D scene which is most commonly a mesh models and rendering of this scene. These two stages will be referred to as modeling and rendering. From this brief description we can already see one problem with using a software like this for creating illustrations of molecular models - they make assumptions about the data. While most of the people using this software would actually be using meshes to represent their 3D scenes, in case of molecular models this is not the case. TODO: say something about "scene graph" which is how scene is implemented usually inside these programs.
Luckily, it has become a convention that all of these programs provide some kind of scripting interface. This enables its users to extend the program with plugins that use the scripting interface. This way the user can implement functionality that he or she is missing from the basic program. Thanks to that these 3D authoring programs are able to be adapted to more use cases. 
Some of these plugins have actually already been implemented to help artists in molecular illustration. The most prominent is Molecular Maya which, as the name suggests, extends Maya with the ability load and manipulate models of macromolecules from Protein Database. Molecular Maya plugins gives its users the ability to load macromolecules based on its PDB ID or locally from pdb file. After the model load, user is able to select the display representation (between TODO: vyjmenovat) and also select with how much detail the model should be rendered.
Similar plugin exists for Blender modeling program as well. It's name is cellblender
This could be a nice reference: https://pdb101.rcsb.org/learn/resource/molecular-animation-q-and-a-interview

On the other side of the spectrum we have custom-built/ad-hoc-built tools coming out as a result of research in rendering and visualization community. These are usually tied to a specific use case, they are a demonstration of a solution. Because of the nature of the data that we are dealing with in molecular visualization, we are able to render our scenes using very effective techniques. The state-of-the-art approach to rendering molecules has been established impostor (or billboard) rendering (TODO: confirm and find references). This technique utilizes the fact that sphere looks the same from all view positions. Because of that we can, instead of rendering sphere consisting of many hundreds of vertices to get the desired level of detail, render sphere using just 4 vertices as a quad always oriented to the camera. This not only optimizes the rendering speed but also memory requirements.

Barely any of these tools is perfect. Some of the tools are very closed and tied to it's creator. The ideal case would be if any new user could approach any of these tools and with a little training be able to use it. This is not the case, unfortunately.
From the ones that are fairly simple from the user experience stand point is Molecular Flipbook[ref]. It provides user the ability to create simple molecular models and animate them with a keyframe-like animation workflow. 

An important part of any scientific illustration are the data one is trying to illustrate. In the case of molecular models, there is cellPACK which helps us with this. cellPACK is a software that assembles large molecular scenes from a description (a recipe) of how this scene should look, what it should contain and in what quantity.

If we only focused on the problem that the iteration loop is too long because of the rendering times, we would be able to find solution in "interactive renderers" - something like OTOY's Octane or whatever is in MODO. These are path tracers that show the intermediate result which enables the artists to see the progress of the renderer rendering the scene. The artist can then see how his actions are reflected in the final render. This solution however would not solve everything for us if we consider that we are rendering molecular scenes. These scene are usually very big and we still need to use different representation other then meshes to represent our scene and use this representation to effectively render the scene. The contribution of this thesis is not only the fact that we shorten the iteration loop time of the artist but also to sketch an idea of how interoperability of molecular programs can be done.

TODO: write about the cooperation between two programs - usually importing/exporting files, what are the standards (there are none, only pdb for molecule data which we take advantage of, maybe describe packing result data file format)

\section{Maya
Autodesk Maya is not a tool for molecular visualization per-se.
\section{cellVIEW}

\message{ !name(thesis.tex) !offset(65) }

\end{document}
